% Usiamo la classe per le presentazioni
\documentclass{beamer}

\usepackage[utf8]{inputenc} % Per gli accenti

% Inseriamo il titolo
\title{Ciao!}
% Inserimento del sottotitolo
\subtitle{Come salutare con ancora più stile}
% La data può essere espressa come ci piace
\date{31/07/2017}
% Autore della presentazione
\author{Davide Polonio}
% Istituzione per cui presentiamo, facoltativa
\institute{Università degli studi di Padova}

% Carichiamo il tema Berlin
\usetheme{Berlin}
% Settiamo il set di colori beaver
\usecolortheme{beaver}

\begin{document}
 \begin{frame}
  % Stampa a titolo automaticamente formattato
  \maketitle
 \end{frame}

 % Iniziamo la sezione su come salutare
 \section{Salutare}
 \begin{frame} % Nuova slide
  \frametitle{Come salutare} % Titolo slide
  Step principali:
  \begin{enumerate} % Elenco puntato
   % Prima di ogni punto mettiamo quando lo
   % vogliamo rendere disponibile
   \item<1-> Alzare la mano
   \item<2-> Rivolgerla all'interessato
   \item<3-> Muoverla a destra e a sinistra
   \item<4-> Fatto!
  \end{enumerate}
 \end{frame}
\end{document}
