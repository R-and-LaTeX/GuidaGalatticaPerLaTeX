% Ringraziamenti
\cleardoublepage
% Creiamo una sezione speciale
\phantomsection
% Non vogliamo nessuno stile
\thispagestyle{empty}
% Segnamo nella lista delle sezioni dei lettori
% PDF questa parte col nome ``ringraziamenti''
\pdfbookmark{Ringraziamenti}{ringraziamenti}

%\bigskip

% Aggiugiamo impostazioni per l'impaginazione
\begingroup
\let\clearpage\relax
\let\cleardoublepage\relax
\let\cleardoublepage\relax

% Creiamo una sezione che non verrà aggiunta all'indice
\section*{Ringraziamenti}
% Ringraziamenti generici, in corsivo
\noindent \textit{Innanzitutto, vorrei esprimere la
mia gratitudine al Prof. Pinco Palla, relatore della
mia tesi, per l'aiuto e il sostegno fornitomi durante la
stesura del lavoro.}\\

\noindent \textit{Desidero ringraziare con affetto i
miei genitori per il sostegno, il grande aiuto e per
essermi stati vicini in ogni momento durante gli anni di
studio.}\\

% Distianziamoci un poco dai ringraziamenti
\bigskip
% Inseriamo la firma composta da posto, data e nome
\noindent\textit{\myLocation, \myTime}
\hfill \myName

\endgroup
