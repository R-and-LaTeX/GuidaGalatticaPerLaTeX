% Cambiamo la numerazione delle pagine
\frontmatter
% Definiamo il frontespizio
\begin{titlepage}
% Settiamolo in posizione centrale
\begin{center}
% Posizioniamo per prima il nome della
% Università, in grassetto e maiuscolo
\begin{LARGE}
\textbf{\myUni}\\
\end{LARGE}

% Inseriamo 10pt, ovvero 10*0.0138pollici
\vspace{10pt}
% Anche qui vogliamo la scritta larga, ma
% senza il corsivo, definiamo il dipartimento
\begin{Large}
\textsc{\myDepartment}\\
\end{Large}
% Altro spazio verticale
\vspace{10pt}
% Iniziamo un'altra parte che sarà in
% maiuscoletto in cui definiamo la facoltà
\begin{large}
\textsc{\myFaculty}\\
\end{large}
% Mettiamo un grande spazio in quando ora
% inseriamo il logo dell'Università
\vspace{30pt}
% Posizioniamo il logo
\begin{figure}[htbp]
\begin{center}
\includegraphics[height=6cm]{logo-unipd}
\end{center}
\end{figure}
% Altro grande spazio verticale per inserire
% il titolo
\vspace{30pt}

% Titolo bello grande, centrato, in grassetto
\begin{LARGE}
\begin{center}
\textbf{\myTitle}\\
\end{center}
\end{LARGE}

% Ci distianziamo dal titolo
\vspace{10pt}
% Tipo di laurea
\begin{large}
\textsl{\myDegree}\\
\end{large}
% Spazio per inserire relatore/laureando
\vspace{40pt}
% Inseriamo sulla sinistra il relatore
\begin{large}
\begin{flushleft}
\textit{Relatore}\\
\vspace{5pt}
Prof. \myProf
\end{flushleft}
% Nella stessa linea
\vspace{0pt}
% Inseriamo il laureando
\begin{flushright}
\textit{Laureando}\\
\vspace{5pt}
\myName
\end{flushright}
\end{large}
% Andiamo a fondo pagina
\vspace{40pt}
% Linea orizzontale
\line(1, 0){338} \\
% Inserimento anno accademico
\begin{normalsize}
\textsc{Anno Accademico \myAA}
\end{normalsize}

\end{center}
\end{titlepage}
