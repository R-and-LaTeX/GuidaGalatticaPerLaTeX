\section{Immagini}
Solitamente in un documento si ha la necessità di inserire delle immagini o 
per spiegare meglio il contenuto o per abbellire il testo. Per inserire le 
immagini in \LaTeX{} solitamente si utlizza il pacchetto \verb!\graphicx! 
(quindi nel preambolo è necessario inserire \verb!\usepackage{graphicx}!). Il 
comando da utilizzare è \verb!\includegraphics[opzioni]{img_path}!, dove le 
opzioni sono opzionali\footnote{No, non ci scusiamo per il gioco di parole} e 
le più comuni sono:
\begin{itemize}
    \item \textbf{width}: per la larghezza;
    \item \textbf{height}: per l'altezza;
    \item \textbf{scale}: per scalare l'immagine (1.0 è il 100\%);
    \item \textbf{keepaspectratio}: per mantenere lo stesso rapporto 
    larghezza-altezza dell’immagine originale
\end{itemize}
\verb!img_path! serve per specificare il percorso dell'immagine da includere, 
che può essere assoluto o relativo (meglio specificare il percorso relativo, 
specificare il percorso assoluto è un po' come passare per casa se devi andare 
dall'aula al bagno). È possibile anche specificare nel preabolo, tramite il 
comando \verb!\graphicspath{{default_path}}! il percorso di default per le 
immagini. In questo modo al posto di un percorso nel comando 
\verb!\includegraphic! è possibile passare il nome dell'immagine voluta.

Come per le tabelle abbiamo bisogno di un ``ambiente'' che contenga le 
immagini, che in questo caso sarà \verb!\begin{figure}...\end{figure}!.

È buona norma accompagnare le immagini con una didascalia che le spieghi. Per 
fare ciò ci viene in aiuto il comando \verb!\caption{...}! che accetta tra le 
parentesi graffe del testo per descrivere le immagini.

Attenzione! Non tutti i formati di immagini sono supportati! I formati 
supporti sono \textbf{pdf}, \textbf{png} e \textbf{jpg}. Nelle nuove 
installazioni di \LaTeX{} è possibile utilizzare però anche il formato \textbf{
eps}.

Di seguito possiamo vedere un esempio di come inserire un'immagine con 
didascalia in un documento \LaTeX{}.
\lstinputlisting{res/examples/esempioimmagine.tex}



\subsection{Affiancare le immagini}
(Fate un respiro profondo che sennò qui si perde il proprio posto in paradiso) 
\\
Per poter affiancare immagini è possibile utilizzare il \textit{subfig} (Non è 
l'unica soluzione, ma è una delle più semplici). Questo permette di creare 
``delle piccole immagini'' all'interno dell'ambiente 
\verb!\begin{figure}...\end{figure}!. Specificando grandezza delle immagini o 
utilizzando i comandi per iniziare una nuova riga. È probabilmente più facile 
con un esempio.
\lstinputlisting{res/examples/esempio4immagini.tex}
Nell'esempio viene utilizzato il comando \verb!\subfloat! per posizionare le 
quattro immagini al quale, tra parentesi quadre, viene specificata la 
didascalia della singola immagine, mentre tra parentesi graffe viene indicata 
l'immagine da mostrare. Con il comando \verb!\caption! viene, infine, 
specificata la didascalia per le immagini raggruppate.

\subsection{Affiancare testo e immagini}
Sì si può anche affiancare testo e immagini. Serve? Bho vedete voi. In ogni 
caso è possibile utilizzando il \textit{package} \verb!wrapfig!. Questo 
funziona in modo analogo al solito modo di inserire le immagini, tranne che 
l'ambiente sarà \verb!\begin{wrapfig}{pos}{dim}...\end{wrapfig}! dove \texttt{
pos} può essere \textbf{r} o \textbf{l} e \texttt{dim} serve per specificare 
le dimensioni di quanto l'immagine deve ``affiancarsi'' al testo. Di seguito un 
esempio. 
\lstinputlisting{res/examples/esempiotestoafianco.tex}


\section{Note a piè di pagina}
Spesso capita di dover mettere delle note a piè di pagina, per esempio per mettere un riferimento o spiegare meglio qualcosa\footnote{O solo per distrarre il lettore, come in questo caso}. Per fare ciò è necessario utilizzare il comando \verb!\footnote{...}! specificando tra graffe il testo della nota. \LaTeX{} in automatico si occuperà della numerazione e del posizionamento.


\section{Collegamenti ipertestuali}

\subsection{Collegamenti nel testo}
Quando si inseriscono delle immagini e se ne fa riferimento nel testo oppure quando si scrive una parte e la si riprende in un'altra sezione del documento si sente la necessità di inserire un collegamento tra le due parti. Per fare ciò è necessario marcare con \verb!\label{...}! a cui si vuole fare riferimento. Tra parentesi graffe un identificativo univoco all'interno del documento. Quando ci si riferisce ad esso, si deve utilizzare \verb!\ref{...}!, specificando tra graffe lo stesso identificativo. È possibile anche fare riferimento alla pagina in cui si trova l'oggetto marcato con \verb!label! utilizzando \verb!\pageref{...}! (e lo stesso identificativo).

\subsubsection{Buone regole per gli identificativi}
Solitamente, agli identificativi utilizzati dalle \verb!label! e \verb!ref! vengono anteposti alcuni caratteri che permettono di identificare che tipo di oggetto è referenziato. 
\begin{table}[]
\centering
\begin{tabular}{ll}
\hline
\textbf{prefisso} & \textbf{tipologia}         \\ \hline
sec:              & section                    \\ \hline
subsec:           & subsection                 \\ \hline
fig:              & figure                     \\ \hline
tab:              & tabella                    \\ \hline
eq:               & formula                    \\ \hline
lst:              & codice                     \\ \hline
itm:              & elementi di lista numerata \\ \hline
alg:              & algoritmo                  \\ \hline
app:              & appendice                  \\ \hline
\end{tabular}
\end{table}

\subsection{Collegamenti al Web}
Il pacchetto che permette l'inserimento in maniera corretta in un documento \LaTeX{} di \textit{link} o \textit{URL} è \verb!hyperref!. Ci sono due modalità principali per inserire queste informazioni. La prima è \verb!\url{indirizzo}!, con la quale verrà mostrato l'URL in formato \textit{monospace}, pensato anche per documenti che devono essere stampati, la seconda è \verb!\href{indirizzo}{testo_da_mostrare}!, la quale non mostrerà esplicitamente l'indirizzo, ma mostrerà solamente il testo voluto (come succede nei siti Web per capirci). 

\subsubsection{Email}
Per inserire gli indirizzi email nel testo è preferibile creare questo comando personalizzato, da inserire nel preambolo del documento \verb!\newcommand{\mail}[1]{\href{mailto:#1}{\texttt{#1}}}! (per il momento fidatevi, spiegheremo in seguito come creare i propri comandi). L'utilizzo è molto semplice, infatti nel documento è sufficiente dare il comando \verb!\email{...}! specificando l'indirizzo di posta elettronica voluto.

\subsubsection{Personalizzazione dei link}
È buona norma, almeno nella creazione dei siti Web, colorare i link con un colore differente dal testo normale, in modo da evidenziare a chi visita la pagina che lì ci si può cliccare. Anche nei documenti non fa male evidenziare questi indirizzi e il pacchetto \verb!hyperref! permette la personalizzazione tramite il comando \verb!\hypersetup{...}! da mettere nel preambolo del documento, al quale devono essere specificate le personalizzazioni volute. Le più importanti sono:
\begin{itemize}
    \item \textbf{colorlinks=false} oppure \textbf{colorlinks=true}, che indica rispettivamente se si vuole che i link vengano riquadrati oppure colorati;
    \item \textbf{urlcolor=color} che serve per specificare il colore dei link (sostituite \textit{color} con un colore in lingua inglese);
\end{itemize}
\href{https://en.wikibooks.org/wiki/LaTeX/Hyperlinks}{Per maggiori info sulla personalizzazione}.