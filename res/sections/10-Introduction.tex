\chapter{Introduzione}

\section{Si mangia?}

No, non ancora. \LaTeX{} (pronunciato come /'latek/ e non /'lateks/) fu ideato 
all'incirca una trentina d'anni fa, nel 1985, da Leslie Lamport (da cui 
\texttt{L} sta per il suo cognome) ed è un linguaggio di 
programmazione\footnote{Per maggiori informazioni: 
\url{https://stackoverflow.com/a/2968527}.} che permette la stesura di documenti 
al fine di generare PDF e altri formati leggibili da umani.

\LaTeX{} non ha un approccio WYSIWYG\footnote{Acronimo per \texttt{What You 
See Is What You Get}, che era evidentemente troppo lungo per essere scritto 
ogni volta.}, ma il codice viene scritto in un simil-\textit{markup} che poi 
viene compilato. È possibile tuttavia aiutarsi con degli IDE per avere aiuti 
durante la stesura del testo.

\section{Se usassimo Word?}

È una domanda che sorge spontanea, e per documenti di piccole dimensioni è 
un'alternativa preferibile. Per documenti grandi come per esempio una tesi o un 
articolo scientifico invece \LaTeX{} vince senza dubbi: permette una facile 
gestione del documento, e permette di gestire cose come la notazione matematica 
e il versionamento in maniera facile.


\begin{table}[H]
\centering
\begin{tabular}{lll}
\hline
\textbf{Uso}          & \textbf{\LaTeX{}} & \textbf{Word} \\
\hline
Piccoli progetti      & \Smiley\Smiley    & \Smiley\Smiley\Smiley  \\
Grandi progetti       & \Smiley\Smiley\Smiley   & \Smiley    \\
Facilità d'uso        & \Smiley\Smiley    & \Smiley\Smiley\Smiley  \\
Curva d'apprendimento & \Smiley     & \Smiley\Smiley\Smiley  \\
Qualità layout        & \Smiley\Smiley\Smiley   & \Smiley\Smiley   \\
Scrittura scientifica & \Smiley\Smiley\Smiley   & \Smiley    \\
\hline
Totale                & 14    & 13   \\
\hline
\end{tabular}
\caption{Confronto tra \LaTeX{} e Word}
\label{my-label}
\end{table}

\section{Da dove abbiamo copiato}

Questa piccola guida non contiene tutto il sapere su \LaTeX{}, ma solo un 
sottoinsieme ristretto: maggiore conoscenza può essere acquisita da:
\begin{itemize}
 \item \href{https://en.wikibooks.org/wiki/LaTeX/}{Wikibook su LaTeX}
 \item \href{http://www.lorenzopantieri.net/LaTeX_files/ArteLaTeX.pdf}{L'arte 
di scrivere in LaTeX}
\end{itemize}

\noindent
Queste fonti sono le stesse da dove gli autori si sono \textit{ispirati}, 
rielaborando materiale anche di altri corsi\footnote{Vedesi le lezioni
\href{http://www.math.unipd.it/~mpolato/didattica/latex/lesson_1.pdf}{1} 
\href{http://www.math.unipd.it/~mpolato/didattica/latex/lesson_2.pdf}{2} e 
\href{http://www.math.unipd.it/~mpolato/didattica/latex/lesson_3.pdf}{3} a 
cura di Mirko Polato.}.
Esistono siti come \href{https://tex.stackexchange.com/}{Stackexchange} che 
possono essere utili per fare domande e ricevere risposte a dubbi che si possono 
avere durante la stesura di un documento. Infine, per chi non masticasse bene 
l'inglese, esiste anche il \href{http://www.guitex.org/}{Gruppo di Utilizzatori 
Italiani di TeX}, che presenta un forum su cui è possibile esprimere dubbi e 
ottenere aiuto.