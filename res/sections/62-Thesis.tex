\section{Creare un proprio stile per la Tesi}
In questa sezione analizzeremo come creare un proprio stile per una Tesi di
laurea. È risaputo infatti come una pubblicazione di questo calibro possa
essere corposa. Come accennato all'inizio della guida, \LaTeX{} è
particolarmente adatto per documenti di certe dimensioni, e qui vedremo come
creare uno stile per la tesi di laurea.

\subsection{Copertina}
Anche se di importante in una tesi è il contenuto, anche l'occhio vuole la sua
parte. La realizzazione di una copertina non è difficile, e qui vedremo come
realizzarla in pochi passi, strutturando il progetto come abbiamo visto nel
capitolo apposito. % TODO mettere un ref per il capitolo 2

Prima di tutto, definiremo dei comandi in maniera tale da rendere la creazione
della copertina più chiara e anche per separare il contenuto dalla
presentazione.

\paragraph*{Creazione delle variabili} Apriamo il nostro file
\texttt{res/config/config.tex} e aggiungiamo le nostre generalità sottoforma di
variabili:

\lstinputlisting{res/examples/tesi/config.tex}

\paragraph*{Creazione del frontespizio} Ora che abbiamo creato le variabili con
le nostre informazioni andiamo a creare la parte grafica, ovvero quella che poi
sarà generata come prima pagina. Per far ciò creiamo un file in
\texttt{res/sections/Frontespizio.tex} e aggiungiamo il seguente contenuto:

\lstinputlisting{res/examples/tesi/frontespizio.tex}

\noindent Ora abbiamo la copertina per la nostra tesi.

\subsection{Seconda pagina, ringraziamenti}


\subsection{Riferimenti bibliografici}
