\chapter{Primi passi}
Possiamo finalmente passare alla creazione del nostro primo documento in \LaTeX{}
!
\lstinputlisting{res/examples/primodocumento.tex}
Non ci sorprenderà molto che, compilando questo file, ci creerà un file PDF con 
all'interno scritto \textit{``Ecco il mio primo documento con \LaTeX{}.''}. Non 
sempre, ahimè, tutto va come vorremmo. Infatti bisogna scrivere le cose in 
maniera corretta altrimenti non verrà prodotto il documento che vogliamo. In 
questo caso per esempio abbiamo un errore.
\lstinputlisting{res/examples/errore.tex}
Sarebbe molto bello che il computer ci dicesse \texttt{Ehi guarda che manca 
una ``u''} ma purtroppo ci dirà 
\verb|! LaTeX Error: \begin{document} ended by \end{docment}.|, meno 
espressivo di certo ma ci aiuta a capire lo stesso dove sbagliamo. Abituatevi! 
Quando si ha a che fare con i linguaggi di programmazione (anche se magari non 
è la vostra aspirazione) non viene mai detto effettivamente cosa c'è che non 
va.

\section{Come è diviso un documento}
Come avete magari già notato nell'esempio, il documento si divide in due parti:
\begin{itemize}
    \item la prima, dove ci sono cose tipo \verb!\documentclass{...}! e 
    \verb!\usepackage{...}! viene detta \textbf{preambolo}. In questa sezione 
    vengono definiti i pacchetti che il documento utilizza (vedremo in 
    seguito) e il tipo di documento (appena dopo, un po' di pazienza!);
    \item la seconda, invece, che inizia e finisce rispettivamente con 
    \verb!\begin{document}! e \verb!\end{document}! viene detta \textbf{corpo} 
    e contiene tutto il testo del documento.
\end{itemize}

\section{Tipi di documento}
\LaTeX{} dà la possibilità di scrivere differenti tipi di documento a seconda 
delle necessità degli utenti. Il tipo di documento è specificato tramite il 
comando \verb!\documentclass! a cui, tra parentesi graffe, viene indicato che 
tipologia si vuole. Le principali tipologie sono:
\begin{itemize}
    \item \textbf{article}: è la tipologia più comune, serve per scrivere 
    piccoli articoli oppure articoli per una rivista;
    \item \textbf{book}: si utilizza per scrivere documenti lunghi (es. la 
    tesi) e libri;
    \item \textbf{report}: anche questa classe è utilizzata per articoli 
    lunghi;
    \item \textbf{letter}: per le lettere;
    \item \textbf{slide}: serve per fare slide, anche se è poco utilizzato;
    \item \textbf{beamer}: anche questo usato per la creazione di slide.
\end{itemize}

\subsection{Esempi}

\subsubsection{Esempio di documento book}
\lstinputlisting{res/examples/esempiobook.tex}

\subsubsection{Esempio di documento report}
\lstinputlisting{res/examples/esempioreport.tex}

\subsubsection{Esempio di documento beamer}
\lstinputlisting{res/examples/esempiobeamer.tex}

\section{Package}
Nella scrittura di un documento, talvolta abbiamo necessità di aggiungere 
alcune cose magiche. In italiano, come in altre lingue a dir la verità, 
abbiamo un sacco di accenti e vorremmo far sì che questi appaiano nel nostro 
documento. Una possibilità per scrivere, per esempio, la \texttt{``è''} è 
\verb!\'e! che però è abbastanza scomoda. In nostro aiuto, però, vengono i 
pacchetti. Possiamo vedere appunto i pacchetti come delle cose magiche scritte 
da altri che ci aiutano a scrivere in maniera più semplice. Vediamo un esempio.
\lstinputlisting{res/examples/accentinoutf8.tex}
\lstinputlisting{res/examples/accentiutf8.tex}
Se proviamo a compilare il primo dei due esempi gli accenti non appariranno, 
mentre nel secondo si, poichè abbiamo utilizzato il pacchetto 
\verb!\usepackage[utf8]{inputenc}!. Questi \textit{packages} ci permettono di 
fare cose ancora più magiche. Provare il seguente documento per credere!
\lstinputlisting{res/examples/cone.tex}
P.S. Noi non vedremo mai cose così magiche.