\chapter{Formule matematiche}
Una delle cose veramente buone di \LaTeX{} è che ci dà la possibilità di 
scrivere le formule matematiche in maniera molto bella... Però c'è una sfilza 
lunghissima di comandi da utilizzare in base alle necessità. Per questo 
vedremo solamente la parte essenziale e per i vari simboli rimanderemo a 
questo indirizzo \url{https://it.wikipedia.org/wiki/
Aiuto:Formule_matematiche_TeX}.

\section{La modalità matematica}
Ci sono due ambienti che ci permettono di scrivere formule matematiche. Il 
primo permette di scrivere formule \textit{inline} e viene aperto e chiuso da 
\verb!$! oppure rispettivamente da \verb!\(! e \verb!\)!. Esempio:
\lstinputlisting{res/examples/mathexample1.tex}
Produrrà questo risultato:\\
La terna pitagorica \`e $x^2+y^2=z^2$ %oppure \(x^2+y^2=z^2\)
\\ \par
Con il secondo ambiente, invece, la formula viene scritta in una nuova linea, 
centrata in mezzo alla pagina. Questo è aperto e chiuso da \verb!$$! oppure 
rispettivamente da \verb!\[! e \verb!\]!. Esempio:
\lstinputlisting{res/examples/mathexample2.tex}
Produrrà questo risultato:\\
La terna pitagorica \`e $$x^2+y^2=z^2$$ %oppure \[x^2+y^2=z^2\]

\section{Apici e pedici}
Spesso nelle formule (vedi gli esempi sopra) abbiamo apici e pedici. Nella 
modalità matematica è possibile inserirli con i caratteri \verb!^! e \verb!_!. 
Il testo che deve essere apice o pedice deve essere specificato tra graffe se 
questo è composto da più di un carattere. Esempio:
\lstinputlisting{res/examples/mathexample3.tex}
Produrrà questo risultato:\\
$$y^{a+b} = x_1 + x_{01}^2$$

\section{Ambienti utili}

\subsection{Equazioni}
L'ambiente specificato con \verb!\begin{equation}...\end{equation}! permette 
di scrivere formule in maniera analoga a \verb!$$!, ma, in più, numera le 
formule.
\lstinputlisting{res/examples/mathexample4.tex}
Produrrà questo risultato:\\
\begin{equation}
    y^{a+b} = x_1 + x_{01}^2
\end{equation}
\par
Se si ha inoltre la necessità di centrare le formule, invece, perché occupano 
più righe è possibile usare l'ambiente \verb!align!. Esempio:
\lstinputlisting{res/examples/mathexample5.tex}
Produrrà questo risultato:\\
\begin{align}
    (a+b)(a-b) &= a^2 - ab + ab + b^2 \\
    &= a^2 - b^2
\end{align}

\subsection{Definizione per casi}
L'ambiente \verb!cases! ci aiuta nel caso si voglia definire, per esempio, una 
funzione per casi. Esempio:
\lstinputlisting{res/examples/mathexample6.tex}
Produrrà questo risultato:\\
$$
    f(x) = 
    \begin{cases}
        1  &\text{se } x > 0 \\
        0  &\text{se } x = 0 \\
        -1 &\text{se } x < 0 \\
    \end{cases}
$$
