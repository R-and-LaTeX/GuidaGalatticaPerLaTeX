\chapter{Creare comandi personalizzati}
Vi ricordate quando parlavamo delle email nella sezione~\ref{subsec:email}? 
Ecco è arrivato il momento di capire come si definiscono i comandi 
personalizzati. Questa possibilità è molto utile per esempio quando abbiamo 
delle cose ripetitive da fare (immaginate per esempio che nella vostra tesi 
dobbiate scrivere sempre una parola in rosso). I comandi personali vengono 
definiti nel preambolo del documento e, per definire un nuovo comando, è 
necessario seguire la seguente sintassi: \\
\begin{center}
    \verb!\newcommand{nome_comando}[argomenti]{definizione}!
\end{center}
Lascio a voi immaginare cosa sia \verb!\nome_comando! e mi concentro sulle 
altre due parti:
\begin{itemize}
    \item \verb!argomenti! è il numero di ``cose'' che vengono passate al 
    comando (come quello che si passa tra parentesi graffe nei normali 
    comandi), il massimo è \textbf{9} (se non viene messo, poichè opzionale, 
    si assume pari a 0);
    \item \verb!definizione! è quello che il nostro comando deve fare. Gli 
    argomenti possono essere sfruttati all'interno del comando tramite 
    \verb!#n! dove \verb!n! è un numero tra 1 e 9, che indica l'argomento 
    appunto.
\end{itemize}

Esempio di un comando senza argomenti:
\lstinputlisting{res/examples/newcommandsenzaarg.tex}

Esempio di un comando con tre argomenti:
\lstinputlisting{res/examples/newcommandconarg.tex}