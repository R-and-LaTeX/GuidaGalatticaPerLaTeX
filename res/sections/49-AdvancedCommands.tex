\chapter{Bibliografia}
Quando si scrive una tesi è importante citare le fonti da cui si sono prese le 
informazioni (se non altro per far vedere che non si stanno dicendo 
stupidaggini). Latex prevede due metodi principali per scrivere la 
bibliografia: uno manuale e uno automatico. Il primo è di gran lunga più 
semplice, il secondo è più personalizzabile e offre alcune comodità in più 
rispetto al primo\footnote{per la bibliografia si faccia riferimento al 
capitolo \texttt{7.2} del libro \textit{L'Arte di scrivere con \LaTeX{}} di 
Lorenzo Pantieri, dove è spiegato molto bene cosa fare in caso di dubbi}.

\section{Bibliografia manuale} \label{sec:bibman}
In questa prima versione della bibliografia i riferimenti bibliografici devono 
essere specificati all'interno dell'ambiente \\
\verb!\begin{thebibliography}{n}...\end{thebibliography}! dove la \texttt{n} 
nelle seconde parentesi graffe deve essere sostituita o dal primo numero 
composto solo da nove che è maggiore del numero di riferimenti (per capirci se 
abbiamo meno di 10 riferimenti si mette 9, se ne abbiamo da 10 fino a 99 si 
mette 99 e così via), nel caso in cui si utilizzino le etichette automatiche, 
o dall'etichette più lunga, nel caso di etichette personalizzate. La specifica 
dei riferimenti funziona in modo molto simile alle liste: infatti, per ogni 
riferimento, si specifica un 
\verb!\bibitem[etichetta_personalizzate]{identificativo}! e di seguito devono 
essere specificate a mano tutte le informazioni necessarie. 
\verb!etichetta_personalizzata! è opzionale e serve se, appunto, si vogliono 
mettere delle etichette che verranno mostrate al posto del testo, 
\verb!identificativo! solitamente segue la sintassi \verb!autore:titolo! (è 
una \textit{best practice}, come nel caso di \verb!\label!). Spesso se il 
titolo del libro è molto lungo, però, viene abbreviato. 

\subsection{Inserire la bibliografia nel documento}
Una volta fatta la bibliografia, saremmo molto felici se questa apparisse 
nel nostro documento e venisse anche mostrata nell'eventuale indice. Per fare 
ciò, \emph{immediatamente prima} di inserire la bibliografia è necessario dare 
i comandi:
\lstinputlisting{res/examples/bibliografiabookreport.tex}
Nel caso di \verb!book! o \verb!report!, mentre è necessario il comando:
\lstinputlisting{res/examples/bibliografiaarticle.tex}
Nel caso di un \verb!article!. Dove\footnote{Parte rubata appunto da 
\textit{L'Arte di scrivere con \LaTeX{}}}:
\begin{itemize}
    \item \verb!\cleardoublepage! fa cominciare la bibliografia in una pagina 
    nuova dispari, assegnando alla voce nell’indice il numero di pagina 
    corretto;
    \item \verb!\clearpage! va dato per assicurare la corretta assegnazione 
    del numero di pagina alla voce nell’indice solo se a fine composizione il 
    corpo del documento terminasse esattamente a fine pagina e la bibliografia 
    cominciasse all’inizio di una pagina nuova (in tal caso si decommenti la 
    riga corrispondente);
    \item \verb!\phantomsection! va dato solo se è caricato anche il 
    \textit{package} \verb!hyperref! (in tal caso si decommenti la riga 
    corrispondente);
    \item \verb!chapter! e \verb!section! indicano il livello della sezione 
    bibliografica (un capitolo e un paragrafo, rispettivamente);
    \item \verb!\bibname! e \verb!\refname! producono nell’indice generale del 
    documento le voci \emph{Bibliografia} e \emph{Riferimenti bibliografici}
    rispettivamente.
\end{itemize}
Esempio:
\lstinputlisting{res/examples/esempiobibliografia.tex}


\section{Bibliografia automatica}
La bibliografia manuale, essendo molto semplica, ha però dei limiti come già 
accennato. Questi sono:
\begin{itemize}
    \item Non viene ordinata in ordine alfabetico;
    \item Tale bibliografia deve essere riscritta per ogni documento;
    \item Se si vuole cambiarne lo stile, bisogna mettere mano ad ogni singola 
    voce!
\end{itemize}
La versione automatica, invece, permette di creare un \textit{database} delle 
opere, riutilizzabile in più documenti. Per andarla a creare si utilizzano i 
\textit{packages} \texttt{biblatex} e \texttt{csquotes}, inserendo nel 
preambolo:
\begin{lstlisting}
\usepackage[autostyle,italian=guillemets]{csquotes}
\usepackage[backend=biber]{biblatex}
\end{lstlisting}

\subsection{Database delle opere}
Per fare il \textit{database} con tutte le opere che cci servono è necessario
creare un file in \texttt{res/bibliography.bib}, e aggiungiamo le nostre fonti
con la seguente sintassi:
\lstinputlisting{res/examples/tesi/bibliografia.bib}
Come possiamo notare il carattere \verb!@! ci permette di indicare il tipo di 
riferimento (articolo, libro, manuale, ...). Questo è seguito da un 
identificativo univoco (vedi la parte dedicata alla sezione~\ref{sec:bibman}) 
e che ogni campo deve seguire la sintassi:
\begin{lstlisting}
<nome campo> = {<contenuto del campo>},
\end{lstlisting}
(ricordarsi la virgola finale, anche all'ultimo campo)\\
I campi possono essere \emph{obbligatori} o \emph{opzionali} a seconda del 
tipo di riferimento (Quelli nell'esempio sono quasi tutti \emph{obbligatori}).
\par Le tipologie di riferimenti nell'esempio sono solamente un piccolo 
sottoinsieme, come i campi che possono essere specificati per ogniuna
\footnote{per un elenco completo consultare al capitolo \texttt{7.2} del libro 
\textit{L'Arte di scrivere con \LaTeX{}}}.
\par Quando si specifica l'autore di un'opera è possibile che:
\begin{enumerate}
    \item Sia necessario specificare più di un autore;
    \item L'autore abbia più cognomi o/e più nomi.
\end{enumerate}
Nel primo caso la sintassi da adottare è:
\begin{lstlisting}
author = {CognomeAutore1, NomeAutore1 and CognomeAutore2, NomeAutore2},
\end{lstlisting}
Cioè specificando un \texttt{and} tra gli autori. Il secondo caso, invece, 
richiede questa sintassi:
\begin{lstlisting}
author = {Cognome1 Cognome2, Nome1 Nome2},
\end{lstlisting}

\subsection{Stile di citazione}
Gli stili che possono essere specificati per le citazioni sono:
\begin{itemize}
    \item \textbf{numeric} Riferimento: numerico ([1], [2], ...);
    \item \textbf{alphabetic} Riferimento: misto (es. [Mor07]);
    \item \textbf{authoryear} Riferimento: autore, anno (es. [Mori,2007]);
    \item \textbf{authortitle} Riferimento: autore, titolo (es. Mori,Titolo).
\end{itemize}
Per scegliere lo stile di citazione è necessario specificarlo quando si importa
il pacchetto \verb!biblatex! in questo modo:
\begin{lstlisting}
\usepackage[style=<stile scelto>, backend=biber]{biblatex}
\end{lstlisting}

\chapter{Creare comandi personalizzati}
Vi ricordate quando parlavamo delle email nella sezione~\ref{subsec:email}? 
Ecco è arrivato il momento di capire come si definiscono i comandi 
personalizzati. Questa possibilità è molto utile per esempio quando abbiamo 
delle cose ripetutive da fare (immaginate per esempio che nella vostra tesi 
dobbiate scrivere sempre una parola in rosso). I comandi personali vengono 
definiti nel preambolo del documento e, per definire un nuovo comando, è 
necessario seguire la seguente sintassi: \\
\begin{center}
    \verb!\newcommand{nome_comando}[argomenti]{definizione}!
\end{center}
Lascio a voi immagionare cosa sia \verb!\nome_comando! e mi concentro sulle 
altre due parti:
\begin{itemize}
    \item \verb!argomenti! è il numero di ``cose'' che vengono passate al 
    comando (come quello che si passa tra parentesi graffe nei normali 
    comandi), il massimo è \textbf{9} (se non viene messo, poichè opzionale, 
    si assume pari a 0);
    \item \verb!definizione! è quello che il nostro comando deve fare. Gli 
    argomenti possono essere sfruttati all'interno del comando tramite 
    \verb!#n! dove \verb!n! è un numero tra 1 e 9, che indica l'argomento 
    appunto.
\end{itemize}

Esempio di un comando senza argomenti:
\lstinputlisting{res/examples/newcommandsenzaarg.tex}

Esempio di un comando con tre argomenti:
\lstinputlisting{res/examples/newcommandconarg.tex}