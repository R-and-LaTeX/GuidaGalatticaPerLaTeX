\chapter{Comandi avanzati}

\section{Bibliografia}
Quando si scrive una tesi è importante citare le fonti da cui si sono prese le 
informazioni (se non altro per far vedere che non si stanno dicendo 
stupidaggini). Latex prevede due metodi principali per scrivere la 
bibliografia: uno manuale e uno automatico. Noi vedremo solamente il primo, 
perchè di gran lunga più semplice\footnote{per la bibliografia automatica si 
rimanda al capitolo \texttt{7.2} del libro \textit{L'Arte di scrivere con 
\LaTeX{}} di Lorenzo Pantieri, dove è spiegato molto bene}.
\par I riferimenti bibliografici devono essere specificati all'interno 
dell'ambiente \\\verb!\begin{thebibliography}{n}...\end{thebibliography}! dove 
la \texttt{n} nelle seconde parentesi graffe deve essere sostituita o dal 
primo numero composto solo da nove che è maggiore del numero di riferimenti 
(per capirci se abbiamo meno di 10 riferimenti si mette 9, se ne abbiamo da 10 
fino a 99 si mette 99 e così via), nel caso in cui si utilizzino le etichette 
automatiche, o dall'etichette più lunga, nel caso di etichette personalizzate. 
La specifica dei riferimenti funziona in modo molto simile alle liste: 
infatti, per ogni riferimento, si specifica un 
\verb!\bibitem[etichetta_personalizzate]{identificativo}! e di seguito devono 
essere specificate a mano tutte le informazioni necessarie. 
\verb!etichetta_personalizzata! è opzionale e serve se, appunto, si vogliono 
mettere delle etichette che verranno mostrate al posto del testo, 
\verb!identificativo! solitamente segue la sintassi \verb!autore:titolo! (è 
una \textit{best practice}, come nel caso di \verb!\label!). Spesso se il 
titolo del libro è molto lungo, però, viene abbreviato. 
\par Una volta fatta la bibliografia, saremmo molto felici se questa apparisse 
nel nostro documento e venisse anche mostrata nell'eventuale indice. Per fare 
ciò, \emph{immediatamente prima} di inserire la bibliografia è necessario dare 
i comandi:
\lstinputlisting{res/examples/bibliografiabookreport.tex}
Nel caso di \verb!book! o \verb!report!, mentre è necessario il comando:
\lstinputlisting{res/examples/bibliografiaarticle.tex}
Nel caso di un \verb!article!. Dove\footnote{Parte rubata appunto da 
\textit{L'Arte di scrivere con \LaTeX{}}}:
\begin{itemize}
    \item \verb!\cleardoublepage! fa cominciare la bibliografia in una pagina 
    nuova dispari, assegnando alla voce nell’indice il numero di pagina 
    corretto;
    \item \verb!\clearpage! va dato per assicurare la corretta assegnazione 
    del numero di pagina alla voce nell’indice solo se a fine composizione il 
    corpo del documento terminasse esattamente a fine pagina e la bibliografia 
    cominciasse all’inizio di una pagina nuova (in tal caso si decommenti la 
    riga corrispondente);
    \item \verb!\phantomsection! va dato solo se è caricato anche il \textit{
    package} \verb!hyperref! (in tal caso si decommenti la riga 
    corrispondente);
    \item \verb!chapter! e \verb!section! indicano il livello della sezione 
    bibliografica (un capitolo e un paragrafo, rispettivamente);
    \item \verb!\bibname! e \verb!\refname! producono nell’indice generale del 
    documento le voci \emph{Bibliografia} e \emph{Riferimenti bibliografici}
    rispettivamente.
\end{itemize}
Esempio:
\lstinputlisting{res/examples/esempiobibliografia.tex}


\section{Creare comandi personalizzati}
Vi ricordate quando parlavamo delle email nella sezione~\ref{subsec:email}? 
Ecco è arrivato il momento di capire come si definiscono i comandi 
personalizzati. Questa possibilità è molto utile per esempio quando abbiamo 
delle cose ripetutive da fare (immaginate per esempio che nella vostra tesi 
dobbiate scrivere sempre una parola in rosso). I comandi personali vengono 
definiti nel preambolo del documento e, per definire un nuovo comando, è 
necessario seguire la seguente sintassi: \\
\begin{center}
    \verb!\newcommand{nome_comando}[argomenti]{definizione}!
\end{center}
Lascio a voi immagionare cosa sia \verb!\nome_comando! e mi concentro sulle 
altre due parti:
\begin{itemize}
    \item \verb!argomenti! è il numero di ``cose'' che vengono passate al 
    comando (come quello che si passa tra parentesi graffe nei normali 
    comandi), il massimo è \textbf{9} (se non viene messo, poichè opzionale, 
    si assume pari a 0);
    \item \verb!definizione! è quello che il nostro comando deve fare. Gli 
    argomenti possono essere sfruttati all'interno del comando tramite 
    \verb!#n! dove \verb!n! è un numero tra 1 e 9, che indica l'argomento 
    appunto.
\end{itemize}

Esempio di un comando senza argomenti:
\lstinputlisting{res/examples/newcommandsenzaarg.tex}

Esempio di un comando con tre argomenti:
\lstinputlisting{res/examples/newcommandconarg.tex}