\chapter{Creare presentazioni: Beamer}

Si, avete letto bene, non vi stiamo prendendo in giro: con \LaTeX{} è possibile
creare anche delle splendide presentazioni. ``Come si fa?'' immagino vi starete
chiedendo, e la risposta è usando un formato di documento
(\verb!\documentclass!) apposito, che si chiama \texttt{beamer}. In questo
capitolo andreamo ad analizzare un attimo il funzionamento di questa classe,
che include dei comandi specifici per la creazione di slide e l'inserimento di
effetti di transizione tra di esse\footnote{\LaTeX{} è in grado di supportare
transizioni anche nei PDF, dando la possibilità di avere un documento leggibile
con un comunle lettore di PDF ma contenente anche effetti, a differenza di
PowerPoint per esempio.}.

\paragraph*{Esempio} Vediamo subito un minuscolo esempio che creerà una slide
contenente un semplice ``Ciao!'':

\lstinputlisting{res/examples/beamerciao.tex}

\noindent Il risultato che otteniamo non è dei più belli. Un punto molto forte
di Beamer però è la sua praticamente infinita possibilità di personalizzazione,
con supporto a temi e a differenti set di colori per ogni tema. Più avanti
vedremo come selezionarli e creare una tabella degli argomenti automaticamente.

\begin{figure}[t]
 \centering
 \includegraphics[scale=0.3]{screenBeamer}
 \caption{Una presentazione in \LaTeX{} personalizzata: è possibile notare come
si possa arrivare ad ottenere un alto livello di personalizzazione tramite i
temi}
 \label{img:screen_beamer}
\end{figure}

\section{I comandi di base}

\subsection{Creare e dare titoli a frame}
Beamer, essendo solo uno stile di documento per \LaTeX{}, eredita tutti i
comandi dello stesso, aggiungendone di nuovi o migliorando quelli che esistono
già. Un comando nuovo che abbiamo visto subito nell'esempio che abbiamo fatto
all'inizio del capitolo è il \verb!\begin{frame} ... \end{frame}!, che ci
permette di creare una nuova slide. Questo comando è ovviamente fondamentale:
senza di esso non possiamo fare alcuna presentazione. Esiste la possibilità di
dare un titolo alla slide che stiamo creando, e questo titolo si va a dare
grazie al comando \verb!\frametitle{ }!, dove all'interno delle parentesi graffe
andremo ad inserire il testo che vogliamo. Quindi se vogliamo dare un titolo
alla nostra slide di saluto potremmo modificare la slide nella maniera seguente:

\lstinputlisting{res/examples/beamerciao2.tex}

\noindent Se state provando a compilare gli esempi presenti nella guida (cosa
che dovreste!) potrete notare cosa in alto a sinistra è successo: una scritta
più grande con colori diversi dal testo è apparsa.

\subsection{Creazione slide iniziale}
Tutte le presentazioni iniziano con una slide iniziale: di solito si mette il
titolo dell'argomento di cui si parlerà seguito da un sottotitolo che
introdurrà l'ascoltatore al tipo di discussione che sta partecipando.
Infine l'autore, l'istituzione (se sta presentando per qualche istituzione) e
la data della presentazione stessa dovrebbero esserci sempre. Un esempio di una
classica slide iniziale lo possiamo vedere in Figura~\ref{img:screen_beamer}.

\paragraph*{Esempio} Vediamo ora come fare dal punto di vista pratico:

\lstinputlisting{res/examples/beamertitolo.tex}

\paragraph*{Analisi dei comandi}
Vediamo un attimo i comandi che sono presenti nell'esempio appena esposto:
\begin{itemize}
 \item \verb!\title{ }! ci permette, senza sorprese, di inserire il titolo del
documento
 \item \verb!\subtitle{ }! anche qui c'è poco da spiegare: questo comando
inserisce il sottotitolo al documento
 \item \verb!\date{ }! fornisce la funzionalità per inserire la data nella
presentazione: è importante notare come la data può essere del formato che più
ci piace, come ad esempio \textit{1/11/1111} oppure \textit{1 novembre
1111}.\footnote{Nonostante la data molto \textit{cool}, si ipotizza che a
quell'epoca le presentazioni al proiettore non andassero per la maggiore.}
 \item \verb!\institute{ }! permette di inserire l'istituzione per cui
presentiamo (o andiamo a presentare)
\end{itemize}

In generale, tutti i campi qui presenti sono facoltativi: possiamo anche non
mettere il titolo, ma con ovvie conseguenze. Inoltre questi campi oltre a
servire per creare la prima pagine del documento possono contribuire a fornire
\textit{metadata}\footnote{\url{https://it.wikipedia.org/wiki/Metadato}} al PDF
che verrà poi generato, per far si che ciò avvenga basterà importare il
pacchetto \texttt{hyperref} con l'opzione \texttt{pdfusetitle}.

\section{Temi}
Applicare un tema ad una presentazione è una buona pratica per creare una
presentazione più piacevole agli ascoltatori, ed è quindi un modo per tenerli
in qualche modo anche attenti a quello che state dicendo\footnote{A meno che
quello che state dicendo non sia terribilmente noioso.}.
Per applicare un tema esistono delle \textit{keyword} che devono essere
utilizzate. L'applicazione di un tema si suddivide in due fasi: selezione del
tema stesso e, se disponibili, selezione di un set di colori possibili per quel
tema. I comandi sono:
\begin{itemize}
 \item \verb!\usetheme{ }! usato per selezionare il tema
 \item \verb!\usecolortheme{ }! usato per selezionare il set di colori per il
tema
\end{itemize}

\subsection{Temi predefiniti}
I temi predefiniti sono comodi quando non si ha troppa voglia di farsene uno, o
quando semplicemente quello che cerchiamo è già pronto. I temi predefiniti che
Beamer offre sono pensati per essere semplici e diretti: non troveremo quindi
stili all'ultimo grido, ma grafici semplici tra cui potremmo scegliere diversi
abbinamenti di colore. Siccome elencare tutti i temi in questo documento non
sarebbe fattibile, invitiamo il nostro affezionato utente ad aprire il proprio
browser preferito ed a puntarlo a questo indirizzo web
\url{http://deic.uab.es/~iblanes/beamer_gallery/}, dove i temi possono essere
esplorati secondo diverse modalità (per tema, per colore oppure per entrambe
le categorie).
Proviamo ora ad applicare un tema alla nostra prima pagina, e vediamo che
succede!

\lstinputlisting{res/examples/beamertema.tex}

\noindent Come possiamo vedere, le nostre slide cominciano a farsi fare sempre
più carine. Provate pure a cambiare i set di colori e il tema a vostro
piacimento, e potrete notare come non dovrete cambiare il contenuto: infatti in
\LaTeX{} i temi non influenzano sulla sostanza della presentazione, come
invece potrebbe accadere in una presentazione PowerPoint per esempio: questo ci
permette di poter cominciare a scrivere contenuto anche se non abbiamo ancora
deciso che aspetto dovrà avere la nostra presentazione.

\section{Transizioni ed effetti}
Una cosa utile durante una presentazione potrebbe essere nascondere parte del
testo all'ascoltatore, in modo da presentare il contenuto di una slide in
maniera controllara, permettendogli di assorbire tutto quello che viene
presentato. Questo obiettivo può essere ottenuto in diverse maniere: si
potrebbero copia-incollare diversi \textit{frame} e aggiungere la parte nuova
ogni volta, ma non è una soluzione pratica. \LaTeX{} infatti fornisce la
possibilità di inserire le transizioni. Noi vedremo come applicarle sugli
elenchi puntati.

\subsection{Transizioni per gli elenchi puntati}
Come abbiamo già detto, mostrare gradualmente un elenco potrebbe essere
utile. Ottenere ciò è super semplice, e si fa ponendo dopo \verb!\item! il
comando \texttt{<n-m>} dove $n$ e $m$ rispettivamente indicano quando deve
apparire quel punto nella slide e quando deve sparire (nota: $m$ è opzionale e
può essere omesso).

\lstinputlisting{res/examples/beamertransizioni.tex}

\noindent Se ora apriamo il PDF e lo mettiamo in modalità presentazione vedremo
come \LaTeX{} ha creato delle slide che scorrendo danno il senso di
transizioni, pemettendoci di far visualizzare il testo che ci interessa in
determinati momenti.
Questi effetti di transizione si possono usare sia con gli elenchi puntati sia
con quelli numerati, il comportamento rimane uguale.
