\chapter{Creare presentazioni: Beamer}

Si, avete letto bene, non vi stiamo prendendo in giro: con \LaTeX{} è possibile
creare anche delle splendide presentazioni. ``Come si fa?'' immagino vi starete
chiedendo, e la risposta è usando un formato di documento
(\verb!\documentclass!) apposito, che si chiama \texttt{beamer}. In questo
capitolo andreamo ad analizzare un attimo il funzionamento di questa classe,
che include dei comandi specifici per la creazione di slide e l'inserimento di
effetti di transizione tra di esse\footnote{\LaTeX{} è in grado di supportare
transizioni anche nei PDF, dando la possibilità di avere un documento leggibile
con un comunle lettore di PDF ma contenente anche effetti, a differenza di
PowerPoint per esempio.}.

\paragraph*{Esempio} Vediamo subito un minuscolo esempio che creerà una slide
contenente un semplice ``Ciao!'':

\lstinputlisting{res/examples/beamerciao.tex}

\noindent Il risultato che otteniamo non è dei più belli. Un punto molto forte
di Beamer però è la sua praticamente infinita possibilità di personalizzazione,
con supporto a temi e a differenti set di colori per ogni tema. Più avanti
vedremo come selezionarli e creare una tabella degli argomenti automaticamente.

\begin{figure}[t]
 \centering
 \includegraphics[scale=0.3]{screenBeamer}
 \caption{Una presentazione in \LaTeX{} personalizzata: è possibile notare come
si possa arrivare ad ottenere un alto livello di personalizzazione tramite i
temi}
 \label{img:screen_beamer}
\end{figure}

\section{I comandi di base}

\paragraph*{Creare e dare titoli a frame}
Beamer, essendo solo uno stile di documento per \LaTeX{}, eredita tutti i
comandi dello stesso, aggiungendone di nuovi o migliorando quelli che esistono
già. Un comando nuovo che abbiamo visto subito nell'esempio che abbiamo fatto
all'inizio del capitolo è il \verb!\begin{frame} ... \end{frame}!, che ci
permette di creare una nuova slide. Questo comando è ovviamente fondamentale:
senza di esso non possiamo fare alcuna presentazione. Esiste la possibilità di
dare un titolo alla slide che stiamo creando, e questo titolo si va a dare
grazie al comando \verb!\frametitle{ }!, dove all'interno delle parentesi graffe
andremo ad inserire il testo che vogliamo. Quindi se vogliamo dare un titolo
alla nostra slide di saluto potremmo modificare la slide nella maniera seguente:

\lstinputlisting{res/examples/beamerciao2.tex}

\noindent Se state provando a compilare gli esempi presenti nella guida (cosa
che dovreste!) potrete notare cosa in alto a sinistra è successo: una scritta
più grande con colori diversi dal testo è apparsa.

\paragraph*{Creazione slide iniziale}
Tutte le presentazioni iniziano con una slide iniziale: di solito si mette il
titolo dell'argomento di cui si parlerà seguito da un sottotitolo che
introdurrà l'ascoltatore al tipo di discussione che sta partecipando.
Infine l'autore, l'istituzione (se sta presentando per qualche istituzione) e
la data della presentazione stessa dovrebbero esserci sempre. Un esempio di una
classica slide iniziale lo possiamo vedere in Figura~\ref{img:screen_beamer}.
