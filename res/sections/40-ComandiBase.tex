\chapter{Comandi Base}

Di seguito verranno illustrati i comandi base di \LaTeX{}, utili per scrivere 
qualsiasi tipo di documento.

\section{Liste}

Le liste nei documenti possono risultare molto utili quando si tratta di 
elencare oggetti in sequenza, in quanto aumentano la leggibilità e 
scorrevolezza di un testo, non appesantendone la lettura.

La lista è composta fondamentalmente da 3 comandi:
\begin{itemize}
 \item \verb!\begin{}! per dichiarare che si sta iniziando una 
lista puntata
 \item \verb!\item! per indicare un punto della lista
 \item \verb!\end{}! per segnalare che la lista puntata è terminata.
\end{itemize}

Esistono due tipologie di liste: elenchi puntati ed elenchi numerati. Per gli 
elenchi \textbf{puntati} si utilizza \texttt{itemize} all'interno delle 
parentesi di \texttt{begin} e \texttt{end}. Se invece si vogliono utilizzare 
gli elenchi \textbf{numerati} la parola da inserire all'interno di 
\texttt{begin} e 
\texttt{end} è \texttt{enumerate}.

\subsection{Esempi}

Per far capire meglio quanto spiegato verranno illustrati un paio di esempi.

\noindent Prima di tutto vediamo una lista puntata:
\lstinputlisting[firstline=0, lastline=6]{res/examples/liste.tex}

\noindent Questo è invece il codice per una lista numerata:
\lstinputlisting[firstline=8, lastline=13]{res/examples/liste.tex}

\subsection{Personalizzare gli elenchi puntati}

Gli elenchi sono interessanti, ma a volte potremmo volere qualcosa di più. 
Fortunatamente \LaTeX{} è stato progettato anche con questa possibilità, ed è 
possibile applicare piccoli trucchetti per rendere i propri elenchi puntati 
``speciali''.

\subsubsection{Item personalizzati}

Gli item possono essere personalizzati e al posto dei punti inserire un simbolo 
o un nome personalizzato. Vediamo subito un esempio.

Scrivendo a esempio il seguente codice:

\lstinputlisting[firstline=16, lastline=21]{res/examples/liste.tex}

\vspace{\abovedisplayskip}
\begin{minipage}{\linewidth}
  Che produrrà:

  \begin{itemize}
    \item[Uno] Questo
    \item[Due] elenco
    \item[Tre] risulta
    \item[Quattro] personalizzato
  \end{itemize} 
 
\end{minipage}
\vspace{\belowdisplayskip}


\noindent Come possiamo notare, non sempre l'aspetto che otteniamo è bello, ma 
questo è un altro discorso.

\subsubsection{Personalizzare gli elenchi puntati}

A volte il classico pallino degli elenchi puntati è proprio brutto, e vorremmo 
sostituirlo con qualcosa di diverso.
Anche questa operazione è piuttosto semplice, e va eseguita nel preambolo.

\noindent Il codice da inserire è il seguente:

\verb!\renewcommand{\labelitemi}{$\ast$}!

\noindent Questo ci modificherà tutti gli elenchi puntati, mettendo al posto 
del classico pallino un asterisco, ``*''. Ci sono altre opzioni, come per 
esempio \verb!$\bullet$!, \verb!$\cdot$! e \verb!$\diamond$!.

\section{Formattazione di base}

Non c'è da sorprendersi che su un linguaggio progettato per scrivere documenti 
esista la formattazione di base.

\paragraph*{Grassetto} \textbf{Per scrivere in grassetto il comando da dare è 
il seguente: }\verb!\textbf{testo da scrivere in grassetto}!.

\paragraph*{Corsivo} \textit{Per il corsivo il comando da dare è invece: }
\verb!\textit{testo in corsivo}!.

\paragraph*{Sottolineato} \underline{Il sottolineato si fa invece con:} 
\verb!\underline{testo sottolineato}!.

\paragraph*{Testo barrato} \sout{Per cancellare il testo: }
\verb!\sout{Testo da cancellare}!\footnote{Nota che per il testo sbarrato è 
necessario aggiungere un altro pacchetto, scrivendo \texttt{\textbackslash 
usepackage[normalem]\{ulem\}}. La parte relativa ai pacchetti verrà trattata 
meglio più avanti.}.

\section{A spasso nella spaziatura}

È anche utile sapere come andare a capo riga nei momenti giusti e anche come 
inserire spazi appositamente nel testo. A differenza di Word dove viene 
commesso il crimine immondo di inserire invii finché non si raggiunge la 
spaziatura desiderata, in \LaTeX{} è possibile inserire quantità di spazio 
precise, compiendo un lavoro più pulito.

\subsection{A capo riga}

Esistono diversi comandi per eseguire la stessa azione. In questo caso, essi 
sono:
\begin{itemize}
 \item \verb!\\!
 \item \verb!\\*!
 \item \verb!\newline!
\end{itemize}

È importante far notare che non è possibile eseguire una cosa del genere in 
\LaTeX{} per andare a capo più volte\footnote{Si, l'abbiamo messo per 
aumentare il numero di pagine di questa guida.}:

\lstinputlisting{res/examples/abominio.tex}

Esiste un comando specifico per fare ciò, che sarà subito spiegato.

\subsection{Inserire pagine}

L'inserimento di una nuova pagina è molto semplice e il comando per creare una 
nuova pagina è \verb!\newpage!.

\subsection{Vuoti cosmici nel testo}

Avere delle parti vuote nel testo può essere utile, si pensi solamente per 
esempio se vogliamo creare una prima pagina diversa, e in generale può servire 
per personalizzare la pagina.

\paragraph*{Spaziatura verticale} Per far ciò esiste una comando magico, 
chiamato \verb!\vspace{ }! che permette di eseguire una spaziatura orizzontale 
del testo. Quantità dello spazio viene misurata solitamente in pixel, quindi ad 
esempio per inserire una spaziatura verticale di 25px sul testo dovrete 
scrivere \verb!\vspace{25px}!. Questo produrrà

\vspace{25px}

\noindent che infine è proprio quello che volevamo (si, sopra abbiamo inserito 
una spaziatura verticale).

\paragraph*{Spaziatura orizzontale} La spaziatura orizzontale si esegue invece 
con il comando \verb!\hspace{ }! e senza sorprese esegue una spaziatura di tipo 
orizzontale.

