\chapter{Comandi Base}

Di seguito verranno illustrati i comandi base di \LaTeX{}, utili per scrivere
qualsiasi tipo di documento.

\section{Liste}

Le liste nei documenti possono risultare molto utili quando si tratta di
elencare oggetti in sequenza, in quanto aumentano la leggibilità e
scorrevolezza di un testo, non appesantendone la lettura.

La lista è composta fondamentalmente da 3 comandi:
\begin{itemize}
 \item \verb!\begin{}! per dichiarare che si sta iniziando una
lista puntata
 \item \verb!\item! per indicare un punto della lista
 \item \verb!\end{}! per segnalare che la lista puntata è terminata.
\end{itemize}

Esistono due tipologie di liste: elenchi puntati ed elenchi numerati. Per gli
elenchi \textbf{puntati} si utilizza \texttt{itemize} all'interno delle
parentesi di \texttt{begin} e \texttt{end}. Se invece si vogliono utilizzare
gli elenchi \textbf{numerati} la parola da inserire all'interno di
\texttt{begin} e
\texttt{end} è \texttt{enumerate}.

\subsection{Esempi}

Per far capire meglio quanto spiegato verranno illustrati un paio di esempi.

\vspace{\abovedisplayskip}
\begin{minipage}{\linewidth}
  \noindent Prima di tutto vediamo una lista puntata:
  \lstinputlisting[firstline=0, lastline=6]{res/examples/liste.tex}
\end{minipage}
\vspace{\belowdisplayskip}

\vspace{\abovedisplayskip}
\begin{minipage}{\linewidth}
  \noindent Questo è invece il codice per una lista numerata:
  \lstinputlisting[firstline=8, lastline=13]{res/examples/liste.tex}
\end{minipage}
\vspace{\belowdisplayskip}


\subsection{Personalizzare gli elenchi puntati}

Gli elenchi sono interessanti, ma a volte potremmo volere qualcosa di più.
Fortunatamente \LaTeX{} è stato progettato anche con questa possibilità, ed è
possibile applicare piccoli trucchetti per rendere i propri elenchi puntati
``speciali''.

\subsubsection{Item personalizzati}

Gli item possono essere personalizzati e al posto dei punti inserire un simbolo
o un nome personalizzato. Vediamo subito un esempio.

Scrivendo a esempio il seguente codice:

\lstinputlisting[firstline=16, lastline=21]{res/examples/liste.tex}

\vspace{\abovedisplayskip}
\begin{minipage}{\linewidth}
  Che produrrà:

  \begin{itemize}
    \item[Uno] Questo
    \item[Due] elenco
    \item[Tre] risulta
    \item[Quattro] personalizzato
  \end{itemize}

\end{minipage}
\vspace{\belowdisplayskip}


\noindent Come possiamo notare, non sempre l'aspetto che otteniamo è bello, ma
questo è un altro discorso.

\subsubsection{Personalizzare gli elenchi puntati}

A volte il classico pallino degli elenchi puntati è proprio brutto, e vorremmo
sostituirlo con qualcosa di diverso.
Anche questa operazione è piuttosto semplice, e va eseguita nel preambolo.

\noindent Il codice da inserire è il seguente:

\verb!\renewcommand{\labelitemi}{$\ast$}!

\noindent Questo ci modificherà tutti gli elenchi puntati, mettendo al posto
del classico pallino un asterisco, ``*''. Ci sono altre opzioni, come per
esempio \verb!$\bullet$!, \verb!$\cdot$! e \verb!$\diamond$!.

\section{Formattazione di base}

Non c'è da sorprendersi che su un linguaggio progettato per scrivere documenti
esista la formattazione di base.

\paragraph*{Grassetto} \textbf{Per scrivere in grassetto il comando da dare è
il seguente: }\verb!\textbf{testo da scrivere in grassetto}!.

\paragraph*{Corsivo} \textit{Per il corsivo il comando da dare è invece: }
\verb!\textit{testo in corsivo}!.

\paragraph*{Sottolineato} \underline{Il sottolineato si fa invece con:}
\verb!\underline{testo sottolineato}!.

\paragraph*{Testo barrato} \sout{Per cancellare il testo: }
\verb!\sout{Testo da cancellare}!\footnote{Nota che per il testo sbarrato è
necessario aggiungere un altro pacchetto, scrivendo \texttt{\textbackslash
usepackage[normalem]\{ulem\}}. La parte relativa ai pacchetti verrà trattata
meglio più avanti.}.

\section{A spasso nella spaziatura}

È anche utile sapere come andare a capo riga nei momenti giusti e anche come
inserire spazi appositamente nel testo. A differenza di Word dove viene
commesso il crimine immondo di inserire invii finché non si raggiunge la
spaziatura desiderata, in \LaTeX{} è possibile inserire quantità di spazio
precise, compiendo un lavoro più pulito.

\subsection{A capo riga}

Esistono diversi comandi per eseguire la stessa azione. In questo caso, essi
sono:
\begin{itemize}
 \item \verb!\\!
 \item \verb!\\*!
 \item \verb!\newline!
\end{itemize}

È importante far notare che non è possibile eseguire una cosa del genere in
\LaTeX{} per andare a capo più volte\footnote{Si, l'abbiamo messo per
aumentare il numero di pagine di questa guida.}:

\lstinputlisting{res/examples/abominio.tex}

Esiste un comando specifico per fare ciò, che sarà subito spiegato.

\subsection{Inserire pagine}

L'inserimento di una nuova pagina è molto semplice e il comando per creare una
nuova pagina è \verb!\newpage!.

\subsection{Vuoti cosmici nel testo}

Avere delle parti vuote nel testo può essere utile, si pensi solamente per
esempio se vogliamo creare una prima pagina diversa, e in generale può servire
per personalizzare la pagina.

\paragraph*{Spaziatura verticale} Per far ciò esiste una comando magico,
chiamato \verb!\vspace{ }! che permette di eseguire una spaziatura orizzontale
del testo. Quantità dello spazio viene misurata solitamente in pixel, quindi ad
esempio per inserire una spaziatura verticale di 25px sul testo dovrete
scrivere \verb!\vspace{25px}!. Questo produrrà

\vspace{25px}

\noindent che infine è proprio quello che volevamo (si, sopra abbiamo inserito
una spaziatura verticale).

\paragraph*{Spaziatura orizzontale} La spaziatura orizzontale si esegue invece
con il comando \verb!\hspace{ }! e senza sorprese esegue una spaziatura di tipo
orizzontale.

\subsection{Riempire i vuoti cosmici}

Ripensandoci, alla fine non è proprio così brutto alle volte riempire quel
freddo spazio bianco che avvinghia i poveri elementi del nostro amato
documento. Siccome siamo a bordo della nostra nave spaziale, è di certo
equipaggiata con degli strumenti che ci permettono di fare ciò. Questi
strumenti in \LaTeX{} si chiamano \texttt{hfill} e \texttt{vfill}, e servono
rispettivamente per riempire spazi orizzontali e verticali. Questa descrizione
sembra un poco povera ed è normale mettere la loro utilità in discussione, ma
vedremo che questi due strumenti si riveleranno utili quando andremo a
manipolare immagini ed ad inserirle.

\newpage
\begin{landscape}

\subsection{Orientare le pagine}

No, non stiamo parlando di come far trovare la posizione alle pagine che si
sono perse, ma vi spiegheremo come sia possibile avere pagine orientate in
maniera orizzontale in un documento con pagine verticali. Vedremo poi come
invece applicare un'orientazione orizzontale ad un intero documento.

\subsubsection{Una pagina orizzontale}

Per avere solo una pagina orizzontale i passi da fare non sono molti, e
consistono a dire al signor \LaTeX{} che vogliamo che la prossima pagina sia
ruotata di 90 gradi.

\vspace{\abovedisplayskip}
\begin{minipage}{\linewidth}
  \noindent Vediamo un esempio completo:
  \lstinputlisting{res/examples/paginaRuotata.tex}
\end{minipage}
\vspace{\belowdisplayskip}

\subsection{Ma io voglio tutto il documento orizzontale!}

E va bene, non c'è bisogno di scaldarsi tanto! Ora vediamo come creare un
documento orientato orizzontalmente. Per farlo ci sono due modi:
\begin{enumerate}
 \item Impostando la rotazione direttamente in \texttt{documentclass}
 \item Impostando la rotazione tramite il pacchetto \texttt{geometry}
\end{enumerate}


\vspace{\abovedisplayskip}
\begin{minipage}{\linewidth}
  \paragraph*{Impostando la rotazione da geometry}

  La rotazione del documento da \texttt{geometry} è impostabile in questa
  maniera:
  \lstinputlisting{res/examples/documentoRuotato.tex}


\end{minipage}
\vspace{\belowdisplayskip}

\end{landscape}
\newpage
\section{Tabelle}

Siamo arrivati ad un argomento spinoso: le tabelle. Come potete immaginare, la
rappresentazione di una tabella in un documento di testo non dev'essere nulla
di semplice. Immaginate con \LaTeX{}, dove c'è la possibilità di specificare le
linee da visualizzare e di poter dire come specificare la formattazione del
testo all'interno di ogni cella: insomma, un vero disastro.
A spezzare una lancia a favore delle tabelle in \LaTeX{} c'è la loro estrema
flessibilità e personalizzazione: vedremo com'è possibile creare tabelle molto
avanzate, senza rischiare che si disintegrino alla prima modifica.

\subsection{Tabelle semplici}

Iniziamo con una tabella semplice. Realizzare una tabella senza troppe pretese
è semplice dopo che si è capito come funziona.

Prima di tutto, per dichiarare che si sta creando una tabella bisogna scrivere
\verb!\begin{table}[ ]!, dove all'interno delle parentesi quadre è possibile
specificare la posizione di dove vogliamo la tabella. Se la vogliamo nella
posizione di dove la stiamo dichiarando scriveremo \verb![H]!\footnote{Nota
che è necessario importare il pacchetto \texttt{float}. Se non volete farlo,
potete scrivere \texttt{h!} che vi darà un risultato simile}.

Dopodiché bisogna dire che vogliamo una tabulazione, e possiamo specificare il
numero di colonne che vogliamo avere. Per fare ciò scriveremo
\verb!\begin{tabular}{}!, dove nel secondo paio di \verb!{ }! specificheremo il
numero di colonne. Vedremo più avanti nel dettaglio questa parte, per il
momento fidatevi.

\vspace{\abovedisplayskip}
\begin{minipage}{\linewidth}
  \subsubsection{Esempietto}

  Vediamo come creare una semplice tabella:
  \lstinputlisting{res/examples/tabellaSemplice.tex}


\end{minipage}
\vspace{\belowdisplayskip}


Che risulterà in questa tabella:
\begin{table}[H]
\centering
\begin{tabular}{llll}
Articoli      & Identificativo & Quantità & Costo \\
Matite        & Cancelleria    & 100      & 20 \euro{}\\
Penne biro    & Cancelleria    & 120      & 15 \euro{}\\
Gomme         & Cancelleria    & 40       & 60 \euro{}\\
Raccoglitori  & Ufficio        & 30       & 12 \euro{}
\end{tabular}
\caption{Una tabella d'esempio}
\label{tabella:esempio}
\end{table}

A questo punto ci piacerebbe però avere dei bordi sulla tabella. Niente di più
facile! Per aggiungere i bordi alla tabella dobbiamo usare \verb!\hline! per le
linee orizzontali, mentre per avere le linee verticali (ovvero tra colonne)
dobbiamo modificare il \verb!\begin{tabular}{llll}! nella seguente
maniera\footnote{Nota che in \LaTeX{} l'utilizzo di linee verticali è a
quanto pare sconsigliato in quanto peggiora le lebbilità della tabella.}:

\vspace{\abovedisplayskip}
\begin{minipage}{\linewidth}
  \subsubsection{Esempietto}

  Vediamo come creare una semplice tabella:
  \lstinputlisting{res/examples/tabellaConBordi.tex}


\end{minipage}
\vspace{\belowdisplayskip}

Che ci darà il seguente risultato:

\begin{table}[H]
\centering
\begin{tabular}{|l|l|l|l|}
\hline
Articoli      & Identificativo & Quantità & Costo \\
\hline
Matite        & Cancelleria    & 100      & 20 \euro{}\\
\hline
Penne biro    & Cancelleria    & 120      & 15 \euro{}\\
\hline
Gomme         & Cancelleria    & 40       & 60 \euro{}\\
\hline
Raccoglitori  & Ufficio        & 30       & 12 \euro{}\\
\hline
\end{tabular}
\caption{Una tabella d'esempio con i bordi}
\label{tabella:esempio2}
\end{table}

Una nota importante da considerare è la possibilità di personalizzare dove
posizionare i bordi o meno: per la vostra curiosità, provate a rimuovere uno o
due \verb!\hline! o qualche sbarretta sull'inizializzazione della tabella, e
vedrete come la tabella si ``evolve''.

Online esistono tool che permettono la creazione di semplici tabelle in
maniera visuale. Sono molto utili e permettono di risparmiare un sacco di
tempo, per cui noi li consigliamo. Un buon sito per la generazione di tabelle
(che permette cose tra cui l'inserimento di bordi in maniera visuale) è
disponibile all'indirizzo \url{http://www.tablesgenerator.com/}.

\subsection{Tabelle su più pagine}

A volte capita di avere una grande mole di dati da mettere in una tabella, e
che questi dati non ci stiano su una pagina. Se avete provato a fare una
tabella più grande di una pagina avrete sicuramente notato di come questa
continui ad andare anche fuori dal termine della pagina stessa, sbattendosene
di qualsiasi bordo presente. Siccome questa \textit{feature} non piaceva a
tutti, qualche programmatore brufoloso e annoiato nel seminterrato di sua madre
si è inventato una bella sera\footnote{L'habitat preferito per un
programmatore.} di implementare le tabelle su più pagine e di pubblicare ciò su
un pacchetto a parte, \texttt{longtable}.
Per usare \texttt{longtable} quindi bisogna includere l'omonimo pacchetto nel
preambolo del documento, con il solito comando a cui siamo abituati, ovvero
\verb!\usepackage{longtable}!.
La creazione di una tabella sarà simile a quella di prima, con qualche piccola
differenza che ora vedremo in questo esempio:
\lstinputlisting{res/examples/tabellaLunga.tex}

Questa risulta in:

\begin{center}
\begin{longtable}{llll}
Articoli      & Identificativo & Quantità & Costo \\
Matite        & Cancelleria    & 100      & 20 \euro{}\\
Penne biro    & Cancelleria    & 120      & 15 \euro{}\\
Gomme         & Cancelleria    & 40       & 60 \euro{}\\
Matite        & Cancelleria    & 100      & 20 \euro{}\\
Penne biro    & Cancelleria    & 120      & 15 \euro{}\\
Gomme         & Cancelleria    & 40       & 60 \euro{}\\
Matite        & Cancelleria    & 100      & 20 \euro{}\\
Penne biro    & Cancelleria    & 120      & 15 \euro{}\\
Gomme         & Cancelleria    & 40       & 60 \euro{}\\
Matite        & Cancelleria    & 100      & 20 \euro{}\\
Penne biro    & Cancelleria    & 120      & 15 \euro{}\\
Gomme         & Cancelleria    & 40       & 60 \euro{}\\
Matite        & Cancelleria    & 100      & 20 \euro{}\\
Penne biro    & Cancelleria    & 120      & 15 \euro{}\\
Gomme         & Cancelleria    & 40       & 60 \euro{}\\
Matite        & Cancelleria    & 100      & 20 \euro{}\\
Penne biro    & Cancelleria    & 120      & 15 \euro{}\\
Gomme         & Cancelleria    & 40       & 60 \euro{}\\
Matite        & Cancelleria    & 100      & 20 \euro{}\\
Penne biro    & Cancelleria    & 120      & 15 \euro{}\\
Gomme         & Cancelleria    & 40       & 60 \euro{}\\
Matite        & Cancelleria    & 100      & 20 \euro{}\\
Penne biro    & Cancelleria    & 120      & 15 \euro{}\\
Gomme         & Cancelleria    & 40       & 60 \euro{}\\
Matite        & Cancelleria    & 100      & 20 \euro{}\\
Penne biro    & Cancelleria    & 120      & 15 \euro{}\\
Gomme         & Cancelleria    & 40       & 60 \euro{}\\
Matite        & Cancelleria    & 100      & 20 \euro{}\\
Penne biro    & Cancelleria    & 120      & 15 \euro{}\\
Gomme         & Cancelleria    & 40       & 60 \euro{}\\
Raccoglitori  & Ufficio        & 30       & 12 \euro{}\\

\caption{Una tabella multipagina d'esempio}
\label{tabella:esempio_tabella_lunga}
\end{longtable}
\end{center}

Le tabelle su più pagine possono diventare estremamente complicate, e quello
che abbiamo presentato è solamente un accenno, giusto per farvi vedere come
funzionano e che esistono.
Alleghiamo qui un esempio di come una tabella può evolvere in qualcosa di
estremamente complicato ma molto personalizzato. Questa è da considerarsi
stregoneria, e per i propositi del nostro corso non verrà spiegata.

\lstinputlisting{res/examples/tabellaStregoneria.tex}

\subsection{Altre informazioni sulle tabelle}

Se pensavate fosse tutto finito qui, vi state sbagliando di grosso. Il signor
\LaTeX{} permette di personalizzare ulteriormente le nostre tabelle, e qui
verranno spiegati alcuni piccoli trucchi che pensiamo possano essere utili a
voi lettori.

\subsubsection{Posizionamento del testo nelle celle}

A volte potrebbe essere utile avere la possibilità di centrare o posizionare il
testo di una tabella in maniera particolare. Questo è ovviamente possibile, e
per farlo bisogna modificare il modo in cui dichiariamo inizialmente la
tabella. Quando mettiamo la dicitura \texttt{l} in
\verb!\begin{tabular}{llll}! stiamo dicendo infatti che vogliamo che il testo
di tutte le celle nella tabella sia orientato a sinistra. Esistono diverse
opzioni per questo comando, che sono \texttt{l}, \texttt{c} e \texttt{r}.

\begin{table}[H]
\centering
\begin{tabular}{|l|l|}
\hline
l & colonne aggiustate a sinistra \\ \hline
c & colonne centrate              \\ \hline
r & colonne aggiustate a destra   \\ \hline
\end{tabular}
\caption{Elenco delle opzioni che si possono mettere nella voce
\texttt{tabular}.}
\label{tab:elenco_opzioni_tabella}
\end{table}

\subsubsection{Impostare manualmente la grandezza delle celle}

Se non viene specificato nessuna impostazione, in \LaTeX{} la larghezza delle
celle viene impostata automaticamente in maniera tale da ottenere il miglior
effetto visibile per il lettore. Sfortunatamente, questo non sempre capita, e a
volte impostare manualmente la larghezza delle colonne può risultare utile.
L'impostazione della larghezza avviene sempre nella linea
\verb!\begin{tabular}{ }!, e al posto di \texttt{l}, \texttt{c} e \texttt{r}
andremo a mettere un altro parametro, che si chiama \texttt{p}.

\paragraph*{Come funziona la keyword \texttt{p}} La \textit{keyword} \texttt{p}
ha bisogno di un parametro per funzionare, e questo parametro è la larghezza in
cm che si vuole della colonna. Vediamo subito un esempio per chiarirci le idee:

\vspace{\abovedisplayskip}
\begin{minipage}{\linewidth}
  \lstinputlisting{res/examples/tabellaSpoporzionata.tex}
\end{minipage}
\vspace{\belowdisplayskip}

\noindent Questo codice ci darà il seguente risultato:

\begin{table}[H]
\centering
\begin{tabular}{|p{2cm}|p{10cm}|}
\hline
È questa colonna sporpozionata? & Si \\ \hline
\end{tabular}
\caption{Una tabella con la larghezza delle colonne personalizzata}
\label{tab:es_spoporzionata}
\end{table}

\noindent Come possiamo notare, bisogna prestare attenzione quando si impostano
manualmente le dimensioni delle varie colonne, perché l'aspetto potrebbe
risentirne (come in questo caso). È buona pratica quindi, se vogliamo impostare
la dimensione delle colonne manualmente, eseguire delle prove visuali in modo
da vedere se il risultato ottenuto ci soddisfa.
