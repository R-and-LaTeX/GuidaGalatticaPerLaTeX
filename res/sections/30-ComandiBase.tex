\chapter{Comandi Base}

Di seguito verranno illustrati i comandi base di \LaTeX{}, utili per scrivere 
qualsiasi tipo di documento.

\section{Formattazione di base}

Non c'è da sorprendersi che su un linguaggio progettato per scrivere documenti 
esista la formattazione di base.

\paragraph*{Grassetto} \textbf{Per scrivere in grassetto il comando da dare è 
il seguente: }\verb!\textbf{testo da scrivere in grassetto}!.

\paragraph*{Corsivo} \textit{Per il corsivo il comando da dare è invece: }
\verb!\textit{testo in corsivo}!.

\paragraph*{Sottolineato} \underline{Il sottolineato si fa invece con:} 
\verb!\underline{testo sottolineato}!.

\paragraph*{Testo barrato} \sout{Per cancellare il testo: }
\verb!\sout{Testo da cancellare}!\footnote{Nota che per il testo sbarrato è 
necessario aggiungere un altro pacchetto, scrivendo \texttt{\textbackslash 
usepackage[normalem]\{ulem\}}. La parte relativa ai pacchetti verrà trattata 
meglio più avanti.}.


\section{Liste}

Le liste nei documenti possono risultare molto utili quando si tratta di 
elencare oggetti in sequenza, in quanto aumentano la leggibilità e 
scorrevolezza di un testo, non appesantendone la lettura.

La lista è composta fondamentalmente da 3 comandi:
\begin{itemize}
 \item \verb!\begin{}! per dichiarare che si sta iniziando una 
lista puntata
 \item \verb!\item! per indicare un punto della lista
 \item \verb!\end{}! per segnalare che la lista puntata è terminata.
\end{itemize}

Esistono due tipologie di liste: elenchi puntati ed elenchi numerati. Per gli 
elenchi \textbf{puntati} si utilizza \texttt{itemize} all'interno delle 
parentesi di \texttt{begin} e \texttt{end}. Se invece si vogliono utilizzare 
gli elenchi \textbf{numerati} la parola da inserire all'interno di 
\texttt{begin} e 
\texttt{end} è \texttt{enumerate}.

\subsection{Esempi}

Per far capire meglio quanto spiegato verranno illustrati un paio di esempi.

% TODO fare esempio lista puntata e numerata.