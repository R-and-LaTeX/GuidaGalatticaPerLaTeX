\chapter{Comandi Base}

Di seguito verranno illustrati i comandi base di \LaTeX{}, utili per scrivere 
qualsiasi tipo di documento.

\section{Liste}

Le liste nei documenti possono risultare molto utili quando si tratta di 
elencare oggetti in sequenza, in quanto aumentano la leggibilità e 
scorrevolezza di un testo, non appesantendone la lettura.

La lista è composta fondamentalmente da 3 comandi:
\begin{itemize}
 \item \verb!\begin{}! per dichiarare che si sta iniziando una 
lista puntata
 \item \verb!\item! per indicare un punto della lista
 \item \verb!\end{}! per segnalare che la lista puntata è terminata.
\end{itemize}

Esistono due tipologie di liste: elenchi puntati ed elenchi numerati. Per gli 
elenchi \textbf{puntati} si utilizza \texttt{itemize} all'interno delle 
parentesi di \texttt{begin} e \texttt{end}. Se invece si vogliono utilizzare 
gli elenchi numerati la parola da inserire all'interno di \texttt{begin} e 
\texttt{end} è \texttt{enumerate}.

\subsection{Esempi}

Per far capire meglio quanto spiegato verranno illustrati un paio di esempi.

% TODO fare esempio lista puntata e numerata.